\documentclass[11pt]{article}

\begin{document}

\section{Development}

\subsection{Build Tool}

The program is implemented using the build tool Cabal (version 3.10.1.0) and GHC (version 9.2.5).
During development, the Haskell Language Server (HSL) (version 1.9.1.0) was used.

\subsubsection{Building}
The executable can be built by invoking the command \verb|cabal build|.
However, installing it with the command \verb|cabal install| is preferable as this approach enables the immediate execution of the program via the CLI.

\subsubsection{Running}

Running the program is possible by invoking \verb|cabal run|, or \verb|MiniCheck| if it was installed using \verb|cabal install|.
Appending \verb|--help| prints the help text.
The program has three modes, which are as follows:

\paragraph{minicheck validate TS\_FILE}
Parse and validate the transition system found at \verb|TS_FILE|.

\paragraph{minicheck ctl TS\_FILE CTL\_FORMULA}
Evaluate the CTL formula in the transition system found at \verb|TS_FILE|.

\paragraph{minicheck ltl TS\_FILE LTL\_FORMULA BOUND}
Evaluate the LTL formula in the transition system found at \verb|TS_FILE| with \verb|BOUND| as the maximum path length.

\subsection{Dependencies}

Beyond the \textit{base} and \textit{containers} libraries, which are usually part of the Haskell prelude, this program uses \textit{cmdargs} (version 0.10.22) and \textit{parsec} (version 3.1.16.1).
The former provides functionality for parsing and validating command line arguments, while the latter's monad parser forms the basis of the program's CTL, LTL, and transition system parsers.

% TODO Haskell Language extensions

\subsection{Testing Framework}

% TODO Testing libraries and frameworks	

\section{Extension}

% TODO Describe implemented extension (LTL model checking)

\section{Modules}

% TODO Describe module partitioning (ctl, ltl, ts)

\section{Testing}

% TODO Describe testing methodology, focus, and potentially untested code
% TODO Describe coverage

\section{Profiling}

% TODO Describe profiling

\section{Known Issues and Limitations}

% TODO Describe known issues and limitations

\end{document}
